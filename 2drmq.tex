
\documentclass{beamer}
\usepackage[polish]{babel}
\usepackage[utf8]{inputenc}
\usepackage[T1]{fontenc}

\usetheme{Montpellier}
\beamersetuncovermixins{\opaqueness<1>{25}}{\opaqueness<2->{15}}

\usecolortheme{dove}
\begin{document}
\title{Dwuwymiarowe Range-Minimum Queries}
\subtitle{Tablice Yuana-Atallaha}
\author{Filip Chudy}
\date{\today}


\begin{frame}
\titlepage
\end{frame}

\section{Wstęp}

\subsection{RMQ -- co to jest?}
\begin{frame} \frametitle{1D RMQ}
 Dana jest jednowymiarowa tablica A rozmiaru N. $A[RMQ_{a,b}(A)] = \min_{a \leq i \leq b} A[i]$.
\end{frame}

\begin{frame} \frametitle{1D RMQ}
 Po ludzku -- w zadanym przedziale szukamy \textbf{pozycji} najmniejszego elementu.
\end{frame}

\begin{frame} \frametitle{Wielowymiarowe RMQ}
 RMQ można uogólnić na więcej wymiarów -- wtedy $N = n_1 \cdot n_2 \cdot \dots \cdot n_d$.
 
 Przedział jest wtedy prostokątny.
\end{frame}


\subsection{Po co to wszystko}
\begin{frame} \frametitle{1D}
 We wszystkich pracach piszą, że RMQ przydaje się w algorytmach tekstowych.
\end{frame}

\begin{frame} \frametitle{2D}
TODO screen z cywilizacji.

 Policzyliśmy wartości wszystkich pól, gdzie mamy postawić miasto?
 
 Innymi słowy -- bot.
\end{frame}

\subsection{O czym powiemy}
\begin{frame} \frametitle{Praca Yuana i Atallaha}
 Rozwiązanie ogólne dla każdej \strong{USTALONEJ} liczby wymiarów.
\end{frame}


\section{1D -- pierwsze kroki}
\subsection{Drzewo przedziałowe}
\begin{frame} \frametitle{Co to?}
 Powtórka z AiSD.
 \pause (BARDZO SŁABE)
\end{frame}

\subsection{Drzewo kartezjańskie}
\begin{frame} \frametitle{Co to?}
 Niech $i = \argmin A$. Drzewo kartezjańskie CT(A):
 \begin{itemize}
  \item ma w korzeniu A[i].
  \item jako lewego syna ma A[1..i-1].
  \item jako prawego syna ma A[i+1..n].
 \end{itemize}
\end{frame}

\begin{frame} \frametitle{Jak je zastosować?}
 
\end{frame}

\begin{frame} \frametitle{Co w 2D?}
 W 1D ratuje nas tani preprocesing przypadków małych.
 
 W 2D jednak...
\end{frame}

\begin{frame} \frametitle{Podsumowanie}
 W przypadku jednowymiarowym drzewa kartezjańskie działają dobrze.
 
 Nie dają się jednak uogólnić na więcej wymiarów.
\end{frame}

\section{Struktura Yuana-Atallaha}

\subsection{Wstęp}
\begin{frame} \frametitle{Wstęp}
 Yuan i Atallah (2010) stworzyli strukturę, która dobrze się uogólnia na więcej wymiarów.
\end{frame}

\begin{frame} \frametitle{Idea}
 
\end{frame}

\begin{frame} \frametitle{Struktura}
 
\end{frame}

\begin{frame} \frametitle{Szkic dowodu}
 Szkic dowodu
\end{frame}
\end{document}