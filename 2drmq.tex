
\documentclass{beamer}
\usepackage[polish]{babel}
\usepackage[utf8]{inputenc}
\usepackage[T1]{fontenc}

\usetheme{Montpellier}
\beamersetuncovermixins{\opaqueness<1>{25}}{\opaqueness<2->{15}}

\usecolortheme{dove}
\begin{document}
\title{Dwuwymiarowe Range-Minimum Queries}
\subtitle{Tablice Yuana-Atallaha}
\author{Filip Chudy}
\date{\today}

\begin{frame}
\titlepage
\end{frame}

\section{Wstęp}
\subsection{RMQ -- co to jest?}
\begin{frame} \frametitle{1D RMQ}
 Dana jest jednowymiarowa tablica A rozmiaru N. $RMQ_{a,b}(A) = \arg\min_{a \leq i \leq b} A[i]$.
\end{frame}

\begin{frame} \frametitle{1D RMQ}
 Po ludzku -- w zadanym przedziale szukamy \textbf{pozycji} najmniejszego elementu.
\end{frame}

\begin{frame} \frametitle{Wielowymiarowe RMQ}
 RMQ można uogólnić na więcej wymiarów.
 
 $N = n_1 \cdot n_2 \cdot \dots \cdot n_d$.
 
 Przedział jest wtedy prostokątny.
\end{frame}

\subsection{Po co to wszystko}
\begin{frame} \frametitle{1D}
 We wszystkich pracach piszą, że RMQ przydaje się w algorytmach tekstowych.
\end{frame}

\begin{frame} \frametitle{2D}
\begin{columns}
\begin{column}{5cm}
\includegraphics[width=\textwidth,height=0.8\textheight,keepaspectratio]{civ.png}
\end{column}
\begin{column}{5cm}
Policzyliśmy wartości funkcji jakości dla wszystkich pól, gdzie mamy postawić miasto?
 
 Innymi słowy -- bot.
\end{column} 
\end{columns} 
\end{frame}

\subsection{O czym powiemy}
\begin{frame} \frametitle{Praca Yuana i Atallaha}
 Rozwiązanie ogólne dla każdej \textbf{USTALONEJ} liczby wymiarów.
\end{frame}


\section{1D -- pierwsze kroki}

\begin{frame} \frametitle{Drzewo kartezjańskie}
 Niech $i = \arg\min A$. Drzewo kartezjańskie CT(A)
 \begin{itemize}
  \item ma w korzeniu A[i].
  \item jako lewego syna ma CT(A[1..i-1]).
  \item jako prawego syna ma CT(A[i+1..n]).
 \end{itemize}
\end{frame}

\begin{frame} \frametitle{Drzewo kartezjańskie}
 W przypadku jednowymiarowym drzewa kartezjańskie działają dobrze.
 
 Nie dają się jednak uogólnić na więcej wymiarów.
\end{frame}

\section{Tablice Yuana-Atallaha (1D)}

\subsection{Wstęp}
\begin{frame} \frametitle{Wstęp}
 Yuan i Atallah (2010) stworzyli strukturę, która dobrze się uogólnia na więcej wymiarów.
\end{frame}

\begin{frame} \frametitle{Osłabienie modelu}
 Drzewa kartezjańskie:
 \begin{enumerate}
  \item Zwracamy dokładne pozycje kolejnych minimów.
  \item Preprocesing w czasie liniowym.
 \end{enumerate}
\end{frame}

\begin{frame} \frametitle{Osłabienie modelu}
 Tablice Y-A:
 \begin{enumerate}
  \item Zwracamy $O(1)$ kandydatów na minimum.
  \item Preprocesing w czasie liniowym.
\end{enumerate}
\end{frame}

\begin{frame} \frametitle{Trochę założeń}
 Bez straty ogólności:
 \begin{itemize}
  \item rozmiar tablicy jest potęgą dwójki,
  \item wszystkie elementy tablicy są różne
 \end{itemize}
\end{frame} 

\begin{frame} \frametitle{W co celujemy ze strukturą}
 Rozmiar -- $O(n\log n)$.
 
 Porównania -- $4n$.
\end{frame}

\subsection{Struktura}
\begin{frame} \frametitle{Przedziały kanoniczne}
 $[k 2^i + 1, (k + 1) 2^i]$
 
 Czyli cała tablica, jej połówki, ćwiartki...
\end{frame}

\begin{frame} \frametitle{Tablice właściwe}
 $LeftMin(I,x) = \min_{i \leq x, i \in I} A[i]$
 
 $RightMin(I,x) = \min_{i \geq x, i \in I} A[i]$
\end{frame}

\begin{frame} \frametitle{Szybkie liczenie tablic}
 Rozważymy tylko LeftMin.
 
 RightMin jest analogiczny.
\end{frame}

\begin{frame} \frametitle{Szybkie liczenie tablic}
 Gdy przedział I jest jednoelementowy, $LeftMin(I,\cdot) = i$.
 
 LeftMin od większych przedziałów będziemy budowali na podstawie wyników dla mniejszych.
\end{frame}

\begin{frame} \frametitle{LeftMin}
 Łączenie $LeftMin(I_1,\cdot)$ i $LeftMin(I_2,\cdot)$ w $LeftMin(I_1 + I_2,\cdot)$.
\end{frame}

\begin{frame} \frametitle{Koszt}
 $Comp(n) = 2 Comp(\frac{n}{2}) + 2 \log n$
 
 Czyli $Comp(n) \leq 4 n$.
\end{frame}

\subsection{Zapytania}
\begin{frame} \frametitle{Jak obsługujemy zapytania?}
 Dla zapytania $q$:
 
 $q \rightarrow SolveQuery([1,n], q)$
\end{frame}

\begin{frame} \frametitle{Jak obsługujemy zapytania?}
 $SolveQuery(I=[u,v], q=[a,b])$

 $a = u \rightarrow LeftMin(I, b)$ \linebreak
 $b = v \rightarrow RightMin(I, a)$\linebreak
 Niech $I = I_1 + I_2$.\linebreak
 $q \subseteq I_1 \rightarrow SolveQuery(I_1, q)$\linebreak
 $q \subseteq I_2 \rightarrow SolveQuery(I_2, q)$\linebreak
 $else \rightarrow \min (RightMin(I_1, a), LeftMin(I_2, b)$.
  
\end{frame}

\begin{frame} \frametitle{Pozbywanie się logarytmu}
Pytamy o przedział $[a,b]$.

Szukany przedział kanoniczny dostaniemy z $LCA(a,b)$.
\end{frame}

\begin{frame} \frametitle{Zróbmy to w RAM-ie}
 Obecne rozwiązanie nadal zużywa $O(n \log n)$.
\end{frame}

\begin{frame} \frametitle{Pocięcie struktury}
 Podzielmy $A$ na rozłączne segmenty $s_i$ rozmiaru $g = c_1 \log n$.
 
 Będzie ich $\frac{n}{g}$.
\end{frame}

\begin{frame} \frametitle{Pocięcie struktury}
 Wyróżnimy dwa rodzaje struktur:
 \begin{itemize}
  \item strukturę ogólną spinającą segmenty,
  \item strukturę lokalną dla każdego segmentu
 \end{itemize}
\end{frame}

\begin{frame} \frametitle{Obsługa zapytań}
 Dla zapytania $q$:
 
 $\exists_i q \subseteq s_i \rightarrow RMQ(s_i, q)$\linebreak
 $\exists_i q \subseteq s_i + s_{i+1} \rightarrow \min (RMQ(s_i, q_1), RMQ(s_{i+1}, q_2))$\linebreak
 $else$ skrajne segmenty jak wyżej, wewnętrzne strukturą ogólną
\end{frame}

\begin{frame} \frametitle{Struktura ogólna}
 Zastosujemy tablice Y-A.
 
 Koszt:$O(\frac{n}{g} \log \frac{n}{g}) = O(n)$.
\end{frame}

\end{document}